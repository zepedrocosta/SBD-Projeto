%%%%%%%%%%%%%%%%%%%%%%%%%%%%%%%%%%%%%%%%%%%%%%%%%%%%%%%%%%%%%%%%%%%%%%%%%%%%%
%                                                                           %
%                            Area for text                                  %
%                                                                           %
%%%%%%%%%%%%%%%%%%%%%%%%%%%%%%%%%%%%%%%%%%%%%%%%%%%%%%%%%%%%%%%%%%%%%%%%%%%%%

\section{Introduction}
\label{sec:intro}

\section{Overview of HammerDB}
\label{sec:hammerdb}

HammerDB is a free, open-source tool for benchmarking the performance of relational databases \cite{enwiki:1275860580}.

It supports popular databases like Oracle, SQL Server, PostgreSQL, MySQL, and more. HammerDB uses industry-standard workloads such as TPROC-C and TPROC-H to simulate real-world database activity.

It offers both a graphical interface and command-line options, making it suitable for developers, DBAs, and system administrators to test, compare, and tune database performance.

\subsection{Overview of TPROC-C}
\label{sec:tproc-c}

TPROC-C is a benchmark designed to evaluate the performance of database management systems (DBMS) using a transactional workload. It simulates a typical online transaction processing (OLTP) environment, focusing on operations like inserts, updates, and deletes across multiple tables.

\subsection{TPROC-C vs TPROC-H}
\label{sec:tproc-c-vs-tproc-h}

TPROC-H is a benchmark designed for data warehousing and analytical workloads, while TPROC-C is focused on transactional processing. TPROC-H emphasizes complex queries and large data sets, whereas TPROC-C simulates real-time transactions with a focus on insert, update, and delete operations.

\section{Problem \& DBMS Summary}
\label{sec:problem}

\section{Benchmark Description}
\label{sec:benchmark}

\section{Methodology}
\label{sec:methodology}

\pagebreak

\subsection{Hardware and Software Setup}
\label{sec:hardware-software-setup}

\begin{table}[h!]
    \centering
    \begin{tabular}{|c|c|c|c|}
        \hline
        \textbf{PC}      & \textbf{1}      & \textbf{2}       & \textbf{3}      \\
        \hline
        \textbf{OS}      & Windows 11      & Windows 11       & Linux (Unraid)  \\
        \hline
        \textbf{CPU}     & Intel i7-13700H & AMD Ryzen 5 3600 & Intel i3-10100F \\
        \hline
        \textbf{Cores}   & 14 (6P 8E)      & 6                & 4               \\
        \hline
        \textbf{Threads} & 20              & 12               & 8               \\
        \hline
        \textbf{RAM}     & 16GB            & 16GB             & 32GB            \\
        \hline
        \textbf{Disk}    & SSD M.2 NVMe    & SSD M.2 NVMe     & SSD M.2 NVMe    \\
        \hline
        \textbf{Read}    & 3500 MB/s       & 2500 MB/s        & 3500 MB/s       \\
        \hline
        \textbf{Write}   & 2700 MB/s       & 2100 MB/s        & 3300 MB/s       \\
        \hline
    \end{tabular}
    \caption{Hardware used in the benchmarks}
    \label{tab:hardware-setup}
\end{table}

On the first PC, the HammerDB server was running on a Ubuntu 22.04 virtual machine with 4 cores and 12GB of RAM. All databases were installed on docker containers on the host machine.

On the third PC, all databases including the HammerDB server were running on docker containers on the host machine.

\section{Results}
\label{sec:results}

\section{Discussion}
\label{sec:discussion}

\section{Conclusions}
\label{sec:conclusions}
